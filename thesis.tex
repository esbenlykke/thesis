% Options for packages loaded elsewhere
\PassOptionsToPackage{unicode}{hyperref}
\PassOptionsToPackage{hyphens}{url}
\PassOptionsToPackage{dvipsnames,svgnames,x11names}{xcolor}
%
\documentclass[
  10pt,
  letterpaper,
  DIV=11,
  numbers=noendperiod]{scrartcl}

\usepackage{amsmath,amssymb}
\usepackage{iftex}
\ifPDFTeX
  \usepackage[T1]{fontenc}
  \usepackage[utf8]{inputenc}
  \usepackage{textcomp} % provide euro and other symbols
\else % if luatex or xetex
  \usepackage{unicode-math}
  \defaultfontfeatures{Scale=MatchLowercase}
  \defaultfontfeatures[\rmfamily]{Ligatures=TeX,Scale=1}
\fi
\usepackage{lmodern}
\ifPDFTeX\else  
    % xetex/luatex font selection
  \setmainfont[]{EB Garamond}
  \setsansfont[]{Libertinus Sans}
\fi
% Use upquote if available, for straight quotes in verbatim environments
\IfFileExists{upquote.sty}{\usepackage{upquote}}{}
\IfFileExists{microtype.sty}{% use microtype if available
  \usepackage[]{microtype}
  \UseMicrotypeSet[protrusion]{basicmath} % disable protrusion for tt fonts
}{}
\makeatletter
\@ifundefined{KOMAClassName}{% if non-KOMA class
  \IfFileExists{parskip.sty}{%
    \usepackage{parskip}
  }{% else
    \setlength{\parindent}{0pt}
    \setlength{\parskip}{6pt plus 2pt minus 1pt}}
}{% if KOMA class
  \KOMAoptions{parskip=half}}
\makeatother
\usepackage{xcolor}
\usepackage[margin=20mm,paperwidth=17cm,paperheight=24cm]{geometry}
\setlength{\emergencystretch}{3em} % prevent overfull lines
\setcounter{secnumdepth}{-\maxdimen} % remove section numbering
% Make \paragraph and \subparagraph free-standing
\ifx\paragraph\undefined\else
  \let\oldparagraph\paragraph
  \renewcommand{\paragraph}[1]{\oldparagraph{#1}\mbox{}}
\fi
\ifx\subparagraph\undefined\else
  \let\oldsubparagraph\subparagraph
  \renewcommand{\subparagraph}[1]{\oldsubparagraph{#1}\mbox{}}
\fi


\providecommand{\tightlist}{%
  \setlength{\itemsep}{0pt}\setlength{\parskip}{0pt}}\usepackage{longtable,booktabs,array}
\usepackage{calc} % for calculating minipage widths
% Correct order of tables after \paragraph or \subparagraph
\usepackage{etoolbox}
\makeatletter
\patchcmd\longtable{\par}{\if@noskipsec\mbox{}\fi\par}{}{}
\makeatother
% Allow footnotes in longtable head/foot
\IfFileExists{footnotehyper.sty}{\usepackage{footnotehyper}}{\usepackage{footnote}}
\makesavenoteenv{longtable}
\usepackage{graphicx}
\makeatletter
\def\maxwidth{\ifdim\Gin@nat@width>\linewidth\linewidth\else\Gin@nat@width\fi}
\def\maxheight{\ifdim\Gin@nat@height>\textheight\textheight\else\Gin@nat@height\fi}
\makeatother
% Scale images if necessary, so that they will not overflow the page
% margins by default, and it is still possible to overwrite the defaults
% using explicit options in \includegraphics[width, height, ...]{}
\setkeys{Gin}{width=\maxwidth,height=\maxheight,keepaspectratio}
% Set default figure placement to htbp
\makeatletter
\def\fps@figure{htbp}
\makeatother

\KOMAoption{captions}{tableheading}
\makeatletter
\makeatother
\makeatletter
\makeatother
\makeatletter
\@ifpackageloaded{caption}{}{\usepackage{caption}}
\AtBeginDocument{%
\ifdefined\contentsname
  \renewcommand*\contentsname{Table of contents}
\else
  \newcommand\contentsname{Table of contents}
\fi
\ifdefined\listfigurename
  \renewcommand*\listfigurename{List of Figures}
\else
  \newcommand\listfigurename{List of Figures}
\fi
\ifdefined\listtablename
  \renewcommand*\listtablename{List of Tables}
\else
  \newcommand\listtablename{List of Tables}
\fi
\ifdefined\figurename
  \renewcommand*\figurename{Figure}
\else
  \newcommand\figurename{Figure}
\fi
\ifdefined\tablename
  \renewcommand*\tablename{Table}
\else
  \newcommand\tablename{Table}
\fi
}
\@ifpackageloaded{float}{}{\usepackage{float}}
\floatstyle{ruled}
\@ifundefined{c@chapter}{\newfloat{codelisting}{h}{lop}}{\newfloat{codelisting}{h}{lop}[chapter]}
\floatname{codelisting}{Listing}
\newcommand*\listoflistings{\listof{codelisting}{List of Listings}}
\makeatother
\makeatletter
\@ifpackageloaded{caption}{}{\usepackage{caption}}
\@ifpackageloaded{subcaption}{}{\usepackage{subcaption}}
\makeatother
\makeatletter
\makeatother
\ifLuaTeX
  \usepackage{selnolig}  % disable illegal ligatures
\fi
\usepackage[style=nature,]{biblatex}
\addbibresource{refs.bib}
\IfFileExists{bookmark.sty}{\usepackage{bookmark}}{\usepackage{hyperref}}
\IfFileExists{xurl.sty}{\usepackage{xurl}}{} % add URL line breaks if available
\urlstyle{same} % disable monospaced font for URLs
\hypersetup{
  colorlinks=true,
  linkcolor={DarkSlateBlue},
  filecolor={Maroon},
  citecolor={DarkSlateBlue},
  urlcolor={DarkRed},
  pdfcreator={LaTeX via pandoc}}

\author{}
\date{}

\begin{document}
\renewcommand*\contentsname{Table of contents}
{
\hypersetup{linkcolor=}
\setcounter{tocdepth}{3}
\tableofcontents
}
\newpage

\hypertarget{preface}{%
\section{Preface}\label{preface}}

The journey of my PhD has been a fulfilling expedition, layered with
explorations, discoveries, struggles, and growth. This endeavor was
fueled by my interest in understanding the objective measurements of
physical behavior and sleep. During my Masters, I found myself
increasingly engrossed in these domains.

My subsequent role as a research assistant at Aarhus University opened
another dimension of learning for me. The works of my peers, employing
machine learning and advanced statistics on accelerometer data,
intrigued me. It was as if I found the nexus of my research interests, a
perfect alignment that seamlessly fused my curiosities and passions.

One of the most significant hurdles was my limited experience with
programming and machine learning, which proved to be a steep learning
curve. However, through persistence, I slowly developed the necessary
skills to analyze and interpret my data effectively. Another major
setback was the failed data collection for my third paper. I spent
months visiting families, mounting an ambulatory PSG device on children
before bedtime, and facing the harsh reality of dealing with poor data
quality.

This thesis brings together my explorations and findings across three
papers that carry a consistent emphasis on improving and validating
methods to leverage accelerometer data for studying human behaviors. The
common thread across these papers is the application of innovative
methods, particularly machine learning techniques, to enhance the
utility, reliability, and accuracy of free-living accelerometer data for
monitoring human sleep and physical activity. The work presented here
constitutes a substantial contribution to the field of sleep and
physical activity research, particularly in the context of large-scale
studies.

Two of these papers have already found their place in peer-reviewed
scientific journals, and the third is under review. All of these works
are included as appendices to this thesis, and their content has been
weaved into the fabric of this thesis.

As I look back at my journey through the PhD program, I am grateful for
this opportunity to delve deep into a subject that I am passionate about
and to contribute to a field that is evolving rapidly. This experience
has instilled in me a sense of tenacity and patience, qualities that I
have come to value deeply. I learned that even the most frustrating
problems have solutions, and the path to those solutions often leads to
personal growth and novel insights.

As I stand on the precipice of my future, I am filled with a sense of
anticipation and excitement for the possibilities that lie ahead. I am
eager to explore new horizons, to encounter new challenges, and to
continue growing as a researcher and as an individual. However, wherever
I go and whatever I do, I will carry with me the memories, experiences,
and lessons from this incredible journey.

These years have shaped me in ways I could never have imagined at the
outset, and for that, I am profoundly grateful. As I close this chapter
of my life, I do so with a sense of accomplishment and a promise of
continued exploration and discovery in my field. After all, every ending
is but a new beginning, and I look forward to the adventures that await.

\newpage

\hypertarget{acknowledgements}{%
\section{Acknowledgements}\label{acknowledgements}}

Throughout this journey, there have been several people who have
influenced, inspired, and supported me. My Main Supervisor, Jan
Christian Brønd, deserves special mention for his guidance and patience.
His commitment to nurturing my development as a researcher and lecturer
has been instrumental. Our collaborative dialogues, be it at the office
or during examinations, have been pivotal in my growth. I also extend my
sincere gratitude to my co-supervisors {[}insert name 2{]} and {[}insert
name 3{]}, and my colleague {[}insert name 4{]}, who have always
provided invaluable insights and perspectives.

Amidst all the academic pursuits, my family remained the cornerstone of
my journey. My wife, the bedrock of our family, kept our home running
smoothly and offered endless support and curiosity about my work. The
joy and love from my four children were my constant sources of
motivation and inspiration.

The PhD journey has taught me the importance of rigour and attention to
detail. My approach to work has been permanently shaped by my experience
as a researcher. The discipline and precision that is required in
research has translated into my everyday life, impacting my approach to
problem-solving, decision-making, and even communication. It's
impressive how research is not merely a vocation but a lens through
which we view the world.

There were also moments of immense joy and satisfaction, like finally
solving a complex analytical problem, having my work accepted for
publication, or simply receiving positive feedback from a student or a
colleague. Those moments fueled my motivation and reminded me of the
importance and impact of my work.

One of the most rewarding aspects of this journey was the opportunity to
be part of an international recognized and experienced research group.
This gave me the chance to work with and learn from some of the most
talented people in my field, to discuss ideas and collaborate on
projects, and to be part of a collective effort to advance knowledge and
understanding in our field.

In retrospect, this PhD journey has been much more than a professional
pursuit. It has been a personal voyage of self-discovery and growth.
Through the highs and lows, the victories and setbacks, the late nights
and early mornings, I've discovered a resilience in myself that I hadn't
known before. I found that I could rise to challenges, learn from
failures, and continue to strive for excellence, no matter the odds.

In closing, I wish to express my deep gratitude for all those who have
supported me throughout this journey - my supervisors, colleagues,
friends, and family. Their faith in my abilities and their constant
encouragement have been my pillars of strength. I hope that the work
presented in this thesis reflects the depth of my dedication and the
extent of my learning journey.

\newpage

\hypertarget{supervisor}{%
\section{Supervisor}\label{supervisor}}

Associate Professor Jan Christian Brønd, PhD

Research Unit for Exercise Epedimiology, Centre of Research in Childhood
Health, Department of Sports Science and Clinical Biomechanics,
University of Southern Denmark, 5230 Odense, Denmark

\hypertarget{assessment-committee}{%
\section{Assessment Committee}\label{assessment-committee}}

\hypertarget{chair}{%
\subsection{Chair}\label{chair}}

Associate Professor Anders And, PhD

\hypertarget{opponents}{%
\subsection{Opponents}\label{opponents}}

Professor Andersine And, PhD

Research Unit for Exercise Epedimiology, Centre of Research in Childhood
Health, Department of Sports Science and Clinical Biomechanics,
University of Southern Denmark, 5230 Odense, Denmark

Associate Professor Fætter Højben, PhD

Research Unit for Exercise Epedimiology, Centre of Research in Childhood
Health, Department of Sports Science and Clinical Biomechanics,
University of Southern Denmark, 5230 Odense, Denmark

\hypertarget{funding}{%
\section{Funding}\label{funding}}

The research presented in this thesis was generously funded by
TrygFonden, under grant numbers ID 130081 and 115606, and by the
European Research Council, under grant number 716657. Additional support
was provided by a one-year scholarship from the Faculty of Health
Sciences.

\newpage

\hypertarget{included-papers}{%
\section{Included Papers}\label{included-papers}}

\hypertarget{paper-i}{%
\subsection{Paper I}\label{paper-i}}

\begin{quote}
Manual Annotation of Time in Bed Using Free-Living Recordings of
Accelerometry Data\autocite{skovgaard_manual_2021}
\end{quote}

published in \href{https://doi.org/10.3390/s21248442}{Sensors}.

The first paper is focused on evaluating the manual annotation to enrich
datasets for use in machine learning of in-bed periods by comparing the
manual annotation method with established EEG-based sleep monitoring
devices and self-reported sleep diaries.

\hypertarget{paper-ii}{%
\subsection{Paper II}\label{paper-ii}}

\begin{quote}
Generalizability and Performance of Methods to Detect Non‑Wear with
Free‑Living Accelerometer
Recordings\autocite{skovgaard_generalizability_2023}
\end{quote}

published in
\href{https://doi.org/10.1038/s41598-023-29666-x}{Scientific Reports}.

The second paper delve into a more specific challenge in physical
activity sensor usage - the detection of non-wear. We propose decision
tree models that combine raw acceleration and skin temperature data to
detect non-wear time and emphasize the importance of external validation
in machine-learned models.

\hypertarget{paper-iii}{%
\subsection{Paper III}\label{paper-iii}}

\begin{quote}
Improving Sleep Quality Estimation: A Comparative Study of Machine
Learning and Deep Learning Techniques Utilizing Free-Living
Accelerometer Data from Thigh-Worn Devices and EEG-Based Sleep Tracking
\end{quote}

Submitted to \href{https://www.nature.com/npjdigitalmed/}{npj Digital
Medicin}.

The third paper focuses on sleep quality estimation using machine
learning and deep learning models. We evaluate these models using data
from thigh-worn accelerometers, presenting a potential alternative for
large-scale sleep studies. We underscore the challenges of classifying
awake periods during in-bed time and the need for further precision in
assessing sleep quality metrics an individual-basis.

\newpage

\hypertarget{english-summary}{%
\section{English Summary}\label{english-summary}}

bla

\newpage

\hypertarget{danish-summary}{%
\section{Danish Summary}\label{danish-summary}}

bla

\newpage

\hypertarget{introduction}{%
\section{Introduction}\label{introduction}}

\hypertarget{outline-of-introduction-section}{%
\subsection{Outline of Introduction
Section}\label{outline-of-introduction-section}}

\hypertarget{overview-and-background}{%
\subsubsection{Overview and Background}\label{overview-and-background}}

\begin{itemize}
\tightlist
\item
  The importance of sleep and physical activity tracking in health
  research.
\item
  The limitations of traditional methods, such as polysomnography and
  self-reported diaries.
\item
  The emergence and potential of wearable accelerometers and machine
  learning models in this field.
\end{itemize}

Throughout the course of a single day, a variety of activities encompass
physical behaviors including sleep, physical activity (PA), and
sedentary behavior\autocite{rolloWholeDayMatters2020}. Over the last
decade, numerous studies have highlighted the unique health advantages
of high PA levels, especially moderate-to-vigorous physical activity
(MVPA)\autocite{krausPhysicalActivityAllCause2019,leeEffectPhysicalInactivity2012},
minimal sedentary periods\autocite{wilmotSedentaryTimeAdults2012}, and
adequate sleep\autocite{cappuccioSleepDurationAllcause2010}. The robust
evidence highlighting the health benefits of optimal sleep and daily
moderate-to-vigorous physical activity (MVPA) has led to the formation
of public guidelines such as the American recommendation of 150 minutes
of MVPA per week\autocite{klPhysicalActivityGuidelines2018}, and the
Danish suggestion of 30 minutes of MVPA per day for
adults\autocite{el-zineFysiskAktivitetVoksne} and 60 minutes per day for
children\autocite{el-zineFysiskAktivitetBorn}. Furthermore, it's advised
that adults get 7-8 hours of
sleep\autocite{consensusconferencepanelRecommendedAmountSleep2015},
whereas children aged 6-12 should sleep 9-12 hours and teenagers aged
13-18 should aim for 8-10 hours of sleep
regularly\autocite{paruthiConsensusStatementAmerican2016}. A growing
body of evidence suggesting a link between high sedentary periods and
negative health outcomes has led to guidelines advocating for decreasing
and interrupting sitting
time\autocite{tremblaySedentaryBehaviorResearch2017}. Traditionally, the
relationship between time spent on each of these behaviors throughout a
24-hour cycle and health outcomes has been studied separately,
neglecting the potential inter-connectedness of these
activities\autocite{rosenberger24HourActivityCycle2019}.

Previous research has shown that, apart from the many health benefits
linked to maintaining sufficient PA levels, it also correlates with
improved sleep duration and
quality\autocite{kredlowEffectsPhysicalActivity2015}. As expected, there
is also evidence suggesting that better sleep may encourage increased
PA\autocite{lambiaseTemporalRelationshipsPhysical2013,mcglincheyPhysicalActivitySleep2014},
with a rising number of studies indicating a connection between these
two
behaviors\autocite{dolezalInterrelationshipSleepExercise2017,chennaouiSleepExerciseReciprocal2015}.
Various studies have observed daily correlations between PA and
sleep\autocite{petteegabrielBidirectionalAssociationsAccelerometerdetermined2017,kishidaIntensiveLongitudinalExamination2016}.
However, these findings have shown inconsistencies, largely attributed
to the varied measurement methodologies. Traditionally, data on time
spent on different physical behaviors has been gathered using
self-reported tools such as questionnaires, diaries, or
interviews\autocite{dowdSystematicLiteratureReview2018}. In addition to
recall and response bias, the challenges of self-reporting include the
difficulty of monitoring multiple behaviors
simultaneously\autocite{rosenberger24HourActivityCycle2019}. As a
result, most published research has generally focused on a single
behavior. Some of these limitations can be addressed with 24-hour
accelerometer-based protocols, especially those worn on the
wrist\autocite{rosenberger24HourActivityCycle2019}. Wrist accelerometers
are small, non-invasive,
waterproof\autocite{welkReliabilityAccelerometrybasedActivity2004},
allowing for continuous, 24-hour wear with minimal disruption to the
wearer, and permit uninterrupted activity measurement throughout the
day, thereby tracking and analyzing daily changes in PA and sleep
behaviors. Furthermore, the latest accelerometers supply raw
acceleration data that can be processed using open-source analytical
techniques to produce estimates of sleep, sedentary behavior, and
PA\autocite{miguelesComparabilityAccelerometerSignal2019}. Simultaneous
measurement of physical behaviors, particularly feasible with wrist
accelerometry, can offer a better comprehension of the influence of
these behaviors on health indicators and their interrelationships. This
data could be critical in shaping future health recommendations or
interventions.

\hypertarget{scope-and-relevance}{%
\paragraph{Scope and Relevance}\label{scope-and-relevance}}

-The need for cost-effective, reliable, and practical alternatives for
large-scale studies. - The potential of free-living accelerometers, and
why they are a compelling subject of study.

\hypertarget{existing-challenges}{%
\paragraph{Existing Challenges}\label{existing-challenges}}

\begin{itemize}
\tightlist
\item
  Discuss the challenges with existing methods, such as identifying
  non-wear time, annotating in-bed periods, and classifying awake
  periods during in-bed time.
\item
  Address the lack of exploration of certain sensor locations, like the
  thigh.
\end{itemize}

\hypertarget{thesis-goals-and-objectives}{%
\paragraph{Thesis Goals and
Objectives}\label{thesis-goals-and-objectives}}

\begin{itemize}
\tightlist
\item
  Clearly state the aim and objectives of your thesis.
\item
  Explain how your thesis will address the identified challenges,
  including improving the manual annotation of in-bed periods, enhancing
  non-wear detection, and estimating sleep quality metrics.
\end{itemize}

\hypertarget{overview-of-the-papers}{%
\paragraph{Overview of the Papers}\label{overview-of-the-papers}}

\begin{itemize}
\item
  Briefly introduce each paper, highlighting the key research question,
  methods, and findings.
\item
  Explain how each paper contributes to your thesis goals and
  objectives.
\end{itemize}

\hypertarget{motivation-for-the-research}{%
\subsubsection{Motivation for the
Research}\label{motivation-for-the-research}}

\hypertarget{the-need-for-improved-annotation-techniques}{%
\paragraph{The Need for Improved Annotation
Techniques}\label{the-need-for-improved-annotation-techniques}}

\begin{itemize}
\tightlist
\item
  Importance of accurate annotation in accelerometer data analysis.
\item
  A brief discussion of the first paper's findings and implications.
\end{itemize}

\hypertarget{improving-non-wear-detection}{%
\paragraph{Improving Non-Wear
Detection}\label{improving-non-wear-detection}}

\begin{itemize}
\tightlist
\item
  Explain the implications of undetected non-wear time on data quality.
\item
  Highlight the findings of your second paper and its relevance.
\end{itemize}

\hypertarget{advancing-sleep-quality-estimation}{%
\paragraph{Advancing Sleep Quality
Estimation}\label{advancing-sleep-quality-estimation}}

\begin{itemize}
\tightlist
\item
  Discuss the impact of sleep quality estimation on understanding human
  sleep behavior.
\item
  Briefly describe the conclusions of your third paper.
\end{itemize}

\hypertarget{methodological-approaches}{%
\subsubsection{Methodological
Approaches}\label{methodological-approaches}}

\begin{itemize}
\item
  Give a brief overview of the methods used across all three studies,
  such as the use of machine learning models, deep learning techniques,
  manual annotation, and decision tree models.
\item
  Explain how these methods address the research objectives and the
  challenges identified earlier.
\end{itemize}

\hypertarget{thesis-structure}{%
\subsubsection{Thesis Structure}\label{thesis-structure}}

Provide an outline of the subsequent chapters of your thesis.

\hypertarget{test-header}{%
\section{Test Header}\label{test-header}}

Utilizing machine learning for sleep and physical activity
identification from accelerometry data is a burgeoning field. This
approach offers potential to model complex non-linear relationships
unattainable with simple statistical methods such as multiple linear or
logistic regression \autocite{fiorillo_automated_2019}. However,
supervised machine learning requires large amounts of accurately
annotated data to ensure accuracy and
generalizability\autocite{van_der_ploeg_modern_2014}.

Sleep is considered a vital element for the overall health and
development of
children\autocite{chaput_systematic_2017,chaput_systematic_2016,st-onge_sleep_2016}.
Healthy sleep is characterized by sufficient duration, appropriate
timing, quality, and absence of disturbances or
disorders\autocite{gruber_position_2014}. Despite its importance, the
use of advanced machine learning techniques to assess sleep measures
from accelerometer data remains
underdeveloped\autocite{haghayegh_application_2020}.

Accelerometry provides an affordable and minimally invasive method to
analyze sleep patterns. Objective measurements are crucial for insights
into individual and population-wide circadian rhythms. Polysomnography
(PSG), the gold standard for objective sleep assessment, records
electroencephalographic (EEG), electromyographic (EMG), and
electrooculographic (EOG) activity. However, PSG is costly and
burdensome due to technician support requirements, intrusive sensor
placements, and overnight monitoring\autocite{vaughn_technical_2008}.

Recent studies have tried to use machine learning techniques for
PSG-assessed sleep-wake classification with wrist
acceleration\autocite{haghayegh_application_2020,sundararajan_sleep_2021,hees_novel_2015}.
Sundararajan et al.\autocite{sundararajan_sleep_2021} used a random
forest machine learning algorithm, achieving an F1 score of 73.9\%.
However, the high false discovery rate indicates limitations in
wrist-worn accelerometers, possibly due to the lack of subject variation
and the use of single-night PSG-recordings.

To overcome these limitations, researchers should increase the number of
subjects and recording days to capture more variation in movement
behavior during sleep. Accelerometry, despite its limitations, is
practical for extended recordings outside of the lab
\autocite{van_de_water_objective_2011}. The focus should be on
developing algorithms for sleep timing rather than sleep staging.

Identifying sleep/wake cycles from accelerometry data requires time in
bed annotation, when participants go to bed and wake up. Although this
isn't actual sleep time, accelerometry is still widely used in sleep
research due to its practical advantages
\autocite{hees_novel_2015,madsen_actigraphy_2013,schwab_actigraphy_2018,barouni_ambulatory_2020}
This annotation could be based on individual sleep diaries, EEG-based
recordings \autocite{younes_staging_2016}, or systems for recording
tracheal sounds
\autocite{dafna_sleep-wake_2015,montazeri_ghahjaverestan_sleepwakefulness_2020}.

This study aims to (1) describe a method for manual annotation of
bedtime and wake-up time with raw accelerometry, (2) evaluate the
accuracy of manual annotation against a single-channel EEG-based sleep
staging system and sleep diary, and (3) assess the inter- and
intra-rater reliability of annotations.

\begin{center}\rule{0.5\linewidth}{0.5pt}\end{center}

The use of body-worn motion sensors, especially accelerometers, to study
human physical activity behavior has gained significant popularity over
recent years, offering an efficient and cost-effective method for
capturing objective movement data
\autocite{dowd_systematic_2018,loyen_sedentary_2017,montoye_raw_2018,migueles_comparability_2019}.
These devices can be worn during various activities with some protocols
allowing detachment during water-based activities, sleep, or certain
sports to prevent injury.

However, such detachment periods, referred to as non-wear time, present
researchers with a challenge in the form of missing data. Non-wear time
can significantly influence the outcomes derived from the acceleration
measurements \autocite{lee_missing_2018}. While some researchers exclude
these periods or attempt to impute the missing data using various
methods, such strategies can introduce bias, particularly when non-wear
times are longer. This issue underscores the importance of accurately
classifying and handling non-wear periods for reliable estimates of
subjects' physical activity behavior during free-living
\autocite{lee_missing_2018}.

The classification of non-wear periods can be acquired by having
subjects keep individual log diaries, although this method can be
cumbersome and potentially error-prone
\autocite{ainsworth_recommendations_2012}. In a bid to reduce subject
burden and enhance accuracy, researchers have employed rule-based
methods and advanced algorithms to classify non-wear time
\autocite{hecht_methodology_2009,ruiz_objectively_2011,troiano_physical_2008}.
Yet, such methods can yield disparate physical activity and sedentary
behavior aggregates depending on non-wear settings, age, and obesity
level, limiting their utility and the ability to make comparisons across
studies \autocite{aadland_comparison_2018,toftager_accelerometer_2013}.

Advancements in accelerometer technology in recent years have enabled
researchers to store raw accelerations, enhancing data granularity and
potentially improving non-wear period classification
\autocite{duncan_wear-time_2018,rasmussen_short-term_2020,zhou_classification_2015}.
However, even with access to higher quantity and quality data, simple
duration-based algorithms carry the risk of falsely misclassifying true
non-wear as inactivity, restricting the ability to detect non-wear
episodes shorter than the designated interval.

Recently, studies have explored the potential of machine learning,
including random forests and deep learning techniques, to classify
non-wear using raw accelerometer data
\autocite[\textcite{syed_novel_2021}]{sundararajan_sleep_2021}. Machine
learning algorithms aim to learn patterns from the training data and
approximate the complex model that best describes the relationship
between the predictors and the outcome. The balance between model
variance and bias is a critical consideration in this process, with the
ultimate goal of optimizing the predictive performance of a
machine-learned model on unseen data.

Despite the performance of models utilizing complex machine learning
algorithms on testing data, their performance on external unseen data
remains largely unknown. This gap in knowledge is often due to a lack of
out-of-distribution data sources and the desire to incorporate all
available data into model training to maximize information capture. Some
studies have utilized surface skin temperature in conjunction with raw
acceleration for non-wear classification
\autocite{duncan_wear-time_2018,zhou_classification_2015}, but the
performance and generalizability of adding surface skin temperature with
advanced machine learning methods have yet to be explored.

Despite technological advancements, there is still room for improvement
in accurately classifying non-wear time in raw accelerometer data. It
prompts the question: what heuristic algorithm or machine-learned model
will perform best on unseen data for classifying non-wear time? To
address this, we created three datasets of raw accelerometer data with
correctly labeled wear- and non-wear time, including surface skin
temperature measurements. Our study aims to (1) train three decision
tree models on accelerometer data from thigh and hip-worn accelerometers
for non-wear time classification and evaluate the importance of surface
skin temperature and minimizing the number of predictors provided to the
model, and (2) assess the performance of machine-learned models and
simple heuristic algorithms across datasets of varying age ranges for
non-wear time classification.

\begin{center}\rule{0.5\linewidth}{0.5pt}\end{center}

A vast body of research highlights the critical role of sleep in
maintaining both mental and physical
health\autocite{ma_sleep_2017,meyer_circadian_2022,k_pavlova_sleep_2019,difrancesco_sleep_2019}.
Consequently, accurate sleep assessment methods are crucial for tracking
sleep patterns and improving our understanding of the sleep-health
relationship. Furthermore, the ease of use and high acceptability of
these methods are essential to facilitate large-scale, longitudinal
studies.

The traditional gold standard for objective sleep measurement,
laboratory-based polysomnography (PSG), has been found to be impractical
in large-scale epidemiological studies due to its high cost, need for
professional administration, and susceptibility to rater
bias\autocite{van_de_water_objective_2011,lee_interrater_2022}. As an
alternative, diaries have been used due to their cost-effectiveness and
simplicity, although they are subject to recall bias and other
limitations\autocite{moore_actigraphy_2015}. An innovative approach
involves device-based measurement methods. These tools, which estimate
sleep duration, are advantageous due to their reduced participant burden
and elimination of potential recall biases. A prominent example of such
tools is body-worn accelerometers, which offer a practical and
affordable means of objectively assessing sleep patterns at home for
extended periods. Accelerometers collect continuous, high-resolution
data for several weeks without requiring recharging, further minimizing
participant burden. Their use in sleep and wake classification began
with a wrist movement-based algorithm developed in 1982, and validated
using PSG\autocite{webster_activity-based_1982}. This algorithm was
refined in 1992\autocite{cole_automatic_1992}, leading to the widely
adopted Cole-Kripke model. With advancements in the field, a variety of
techniques, including heuristic algorithms, machine learning models,
regression, and deep learning, are now used to analyze data from hip and
wrist-worn
accelerometers\autocite{palotti_benchmark_2019,cole_automatic_1992,sazonov_activity-based_2004,sadeh_activity-based_1994,hees_novel_2015,sundararajan_sleep_2021}.

While wrist and hip-worn devices have benefited from extensive
methodological development, thigh-worn accelerometers have not seen the
same level of advancement. Existing studies mainly focus on
distinguishing sleep from wakefulness, with emphasis on defining `waking
time' and `bedtime'
\autocite{carlson_validity_2021,inan-eroglu_comparison_2021,van_der_berg_identifying_2016,winkler_identifying_2016}.
Recent strides in estimating sleep duration using thigh-worn devices
have been made, including the introduction of a promising algorithm and
its comparison against PSG\autocite{johansson_development_2023}. Despite
these advancements, the application of machine learning techniques in
this area is still unexplored. Considering the potential of thigh-worn
accelerometers for accurate physical behavior
assessment\autocite{skotte_detection_2014,arvidsson_re-examination_2019},
there is a significant research gap. Therefore, future studies need to
develop techniques similar to those used for wrist and hip-worn
accelerometers, with the ultimate goal of establishing a more holistic,
accurate, and user-friendly method of sleep and physical activity
tracking.

The Zmachine®️ Insight+ (ZM) emerges as a valuable tool within this
landscape. Favorably validated against
PSG\autocite{kaplan_performance_2014,wang_evaluation_2015}, the ZM
provides comparable data without the high costs or the need for
professional monitoring typically associated with PSG. Crucially, the ZM
facilitates multi-night analysis in free-living conditions due to its
ease of use\autocite{pedersen_self-administered_2021}, capturing the
natural variations in sleep patterns. This makes it advantageous over
single-night PSG, particularly as a gold standard data source in machine
learning tasks, as it provides multiple nights of measurements without
inter-rater bias. Despite these benefits, the ZM, like PSG, still poses
a significant participant burden and cost, reinforcing the need for more
accessible alternatives like accelerometers.

Our primary objective in this study was to evaluate a range of machine
learning and deep learning models, utilizing the raw data collected from
a tri-axial thigh-worn accelerometer to estimate in-bed and sleep time.
To ensure the reliability and effectiveness of our models, we compared
their outputs with an electroencephalography-based (EEG) sleep tracking
device, which we, in this current study, considered as the gold standard
for measuring sleep. Furthermore, our secondary goal was to assess the
developed models' performance in evaluating important sleep quality
metrics, including sleep period time (SPT), total sleep time (TST),
sleep efficiency (SE), latency until persistent sleep (LPS), and wake
after sleep onset (WASO).

\newpage


\printbibliography[title=References]


\end{document}
