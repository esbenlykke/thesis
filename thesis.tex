% Options for packages loaded elsewhere
\PassOptionsToPackage{unicode}{hyperref}
\PassOptionsToPackage{hyphens}{url}
\PassOptionsToPackage{dvipsnames,svgnames,x11names}{xcolor}
%
\documentclass[
  8pt,
  letterpaper,
  DIV=11,
  numbers=noendperiod]{scrartcl}

\usepackage{amsmath,amssymb}
\usepackage{iftex}
\ifPDFTeX
  \usepackage[T1]{fontenc}
  \usepackage[utf8]{inputenc}
  \usepackage{textcomp} % provide euro and other symbols
\else % if luatex or xetex
  \usepackage{unicode-math}
  \defaultfontfeatures{Scale=MatchLowercase}
  \defaultfontfeatures[\rmfamily]{Ligatures=TeX,Scale=1}
\fi
\usepackage{lmodern}
\ifPDFTeX\else  
    % xetex/luatex font selection
  \setmainfont[]{EB Garamond}
  \setsansfont[]{Libertinus Sans}
\fi
% Use upquote if available, for straight quotes in verbatim environments
\IfFileExists{upquote.sty}{\usepackage{upquote}}{}
\IfFileExists{microtype.sty}{% use microtype if available
  \usepackage[]{microtype}
  \UseMicrotypeSet[protrusion]{basicmath} % disable protrusion for tt fonts
}{}
\makeatletter
\@ifundefined{KOMAClassName}{% if non-KOMA class
  \IfFileExists{parskip.sty}{%
    \usepackage{parskip}
  }{% else
    \setlength{\parindent}{0pt}
    \setlength{\parskip}{6pt plus 2pt minus 1pt}}
}{% if KOMA class
  \KOMAoptions{parskip=half}}
\makeatother
\usepackage{xcolor}
\usepackage[margin=20mm,paperwidth=17cm,paperheight=24cm]{geometry}
\setlength{\emergencystretch}{3em} % prevent overfull lines
\setcounter{secnumdepth}{-\maxdimen} % remove section numbering
% Make \paragraph and \subparagraph free-standing
\ifx\paragraph\undefined\else
  \let\oldparagraph\paragraph
  \renewcommand{\paragraph}[1]{\oldparagraph{#1}\mbox{}}
\fi
\ifx\subparagraph\undefined\else
  \let\oldsubparagraph\subparagraph
  \renewcommand{\subparagraph}[1]{\oldsubparagraph{#1}\mbox{}}
\fi


\providecommand{\tightlist}{%
  \setlength{\itemsep}{0pt}\setlength{\parskip}{0pt}}\usepackage{longtable,booktabs,array}
\usepackage{calc} % for calculating minipage widths
% Correct order of tables after \paragraph or \subparagraph
\usepackage{etoolbox}
\makeatletter
\patchcmd\longtable{\par}{\if@noskipsec\mbox{}\fi\par}{}{}
\makeatother
% Allow footnotes in longtable head/foot
\IfFileExists{footnotehyper.sty}{\usepackage{footnotehyper}}{\usepackage{footnote}}
\makesavenoteenv{longtable}
\usepackage{graphicx}
\makeatletter
\def\maxwidth{\ifdim\Gin@nat@width>\linewidth\linewidth\else\Gin@nat@width\fi}
\def\maxheight{\ifdim\Gin@nat@height>\textheight\textheight\else\Gin@nat@height\fi}
\makeatother
% Scale images if necessary, so that they will not overflow the page
% margins by default, and it is still possible to overwrite the defaults
% using explicit options in \includegraphics[width, height, ...]{}
\setkeys{Gin}{width=\maxwidth,height=\maxheight,keepaspectratio}
% Set default figure placement to htbp
\makeatletter
\def\fps@figure{htbp}
\makeatother

\KOMAoption{captions}{tableheading}
\makeatletter
\makeatother
\makeatletter
\makeatother
\makeatletter
\@ifpackageloaded{caption}{}{\usepackage{caption}}
\AtBeginDocument{%
\ifdefined\contentsname
  \renewcommand*\contentsname{Table of contents}
\else
  \newcommand\contentsname{Table of contents}
\fi
\ifdefined\listfigurename
  \renewcommand*\listfigurename{List of Figures}
\else
  \newcommand\listfigurename{List of Figures}
\fi
\ifdefined\listtablename
  \renewcommand*\listtablename{List of Tables}
\else
  \newcommand\listtablename{List of Tables}
\fi
\ifdefined\figurename
  \renewcommand*\figurename{Figure}
\else
  \newcommand\figurename{Figure}
\fi
\ifdefined\tablename
  \renewcommand*\tablename{Table}
\else
  \newcommand\tablename{Table}
\fi
}
\@ifpackageloaded{float}{}{\usepackage{float}}
\floatstyle{ruled}
\@ifundefined{c@chapter}{\newfloat{codelisting}{h}{lop}}{\newfloat{codelisting}{h}{lop}[chapter]}
\floatname{codelisting}{Listing}
\newcommand*\listoflistings{\listof{codelisting}{List of Listings}}
\makeatother
\makeatletter
\@ifpackageloaded{caption}{}{\usepackage{caption}}
\@ifpackageloaded{subcaption}{}{\usepackage{subcaption}}
\makeatother
\makeatletter
\makeatother
\ifLuaTeX
  \usepackage{selnolig}  % disable illegal ligatures
\fi
\usepackage[style=nature,]{biblatex}
\addbibresource{refs.bib}
\IfFileExists{bookmark.sty}{\usepackage{bookmark}}{\usepackage{hyperref}}
\IfFileExists{xurl.sty}{\usepackage{xurl}}{} % add URL line breaks if available
\urlstyle{same} % disable monospaced font for URLs
\hypersetup{
  colorlinks=true,
  linkcolor={DarkSlateBlue},
  filecolor={Maroon},
  citecolor={DarkSlateBlue},
  urlcolor={DarkRed},
  pdfcreator={LaTeX via pandoc}}

\author{}
\date{}

\begin{document}
\renewcommand*\contentsname{Table of contents}
{
\hypersetup{linkcolor=}
\setcounter{tocdepth}{3}
\tableofcontents
}
\newpage

\hypertarget{preface}{%
\section{Preface}\label{preface}}

The journey of my PhD has been a fulfilling expedition, layered with
explorations, discoveries, struggles, and growth. This endeavor was
fueled by my interest in understanding the objective measurements of
physical behavior and sleep. During my Masters, I found myself
increasingly engrossed in these domains.

My subsequent role as a research assistant at Aarhus University opened
another dimension of learning for me. The works of my peers, employing
machine learning and advanced statistics on accelerometer data,
intrigued me. It was as if I found the nexus of my research interests, a
perfect alignment that seamlessly fused my curiosities and passions.

One of the most significant hurdles was my limited experience with
programming and machine learning, which proved to be a steep learning
curve. However, through persistence, I slowly developed the necessary
skills to analyze and interpret my data effectively. Another major
setback was the failed data collection for my third paper. I spent
months visiting families, mounting an ambulatory PSG device on children
before bedtime, and facing the harsh reality of dealing with poor data
quality.

This thesis brings together my explorations and findings across three
papers that carry a consistent emphasis on improving and validating
methods to leverage accelerometer data for studying human behaviors. The
common thread across these papers is the application of innovative
methods, particularly machine learning techniques, to enhance the
utility, reliability, and accuracy of free-living accelerometer data for
monitoring human sleep and physical activity. The work presented here
constitutes a substantial contribution to the field of sleep and
physical activity research, particularly in the context of large-scale
studies.

Two of these papers have already found their place in peer-reviewed
scientific journals, and the third is under review. All of these works
are included as appendices to this thesis, and their content has been
weaved into the fabric of this thesis.

As I look back at my journey through the PhD program, I am grateful for
this opportunity to delve deep into a subject that I am passionate about
and to contribute to a field that is evolving rapidly. This experience
has instilled in me a sense of tenacity and patience, qualities that I
have come to value deeply. I learned that even the most frustrating
problems have solutions, and the path to those solutions often leads to
personal growth and novel insights.

As I stand on the precipice of my future, I am filled with a sense of
anticipation and excitement for the possibilities that lie ahead. I am
eager to explore new horizons, to encounter new challenges, and to
continue growing as a researcher and as an individual. However, wherever
I go and whatever I do, I will carry with me the memories, experiences,
and lessons from this incredible journey.

These years have shaped me in ways I could never have imagined at the
outset, and for that, I am profoundly grateful. As I close this chapter
of my life, I do so with a sense of accomplishment and a promise of
continued exploration and discovery in my field. After all, every ending
is but a new beginning, and I look forward to the adventures that await.

\newpage

\hypertarget{acknowledgements}{%
\section{Acknowledgements}\label{acknowledgements}}

Throughout this journey, there have been several people who have
influenced, inspired, and supported me. My Main Supervisor, {[}insert
name 1{]}, deserves special mention for his guidance and patience. His
commitment to nurturing my development as a researcher and lecturer has
been instrumental. Our collaborative dialogues, be it at the office or
during examinations, have been pivotal in my growth. I also extend my
sincere gratitude to my co-supervisors {[}insert name 2{]} and {[}insert
name 3{]}, and my colleague {[}insert name 4{]}, who have always
provided invaluable insights and perspectives.

Amidst all the academic pursuits, my family remained the cornerstone of
my journey. My wife, the bedrock of our family, kept our home running
smoothly and offered endless support and curiosity about my work. The
joy and love from my four children were my constant sources of
motivation and inspiration.

The PhD journey has taught me the importance of rigour and attention to
detail. My approach to work has been permanently shaped by my experience
as a researcher. The discipline and precision that is required in
research has translated into my everyday life, impacting my approach to
problem-solving, decision-making, and even communication. It's
impressive how research is not merely a vocation but a lens through
which we view the world.

There were also moments of immense joy and satisfaction, like finally
solving a complex analytical problem, having my work accepted for
publication, or simply receiving positive feedback from a student or a
colleague. Those moments fueled my motivation and reminded me of the
importance and impact of my work.

One of the most rewarding aspects of this journey was the opportunity to
be part of an international recognized and experienced research group.
This gave me the chance to work with and learn from some of the most
talented people in my field, to discuss ideas and collaborate on
projects, and to be part of a collective effort to advance knowledge and
understanding in our field.

In retrospect, this PhD journey has been much more than a professional
pursuit. It has been a personal voyage of self-discovery and growth.
Through the highs and lows, the victories and setbacks, the late nights
and early mornings, I've discovered a resilience in myself that I hadn't
known before. I found that I could rise to challenges, learn from
failures, and continue to strive for excellence, no matter the odds.

In closing, I wish to express my deep gratitude for all those who have
supported me throughout this journey - my supervisors, colleagues,
friends, and family. Their faith in my abilities and their constant
encouragement have been my pillars of strength. I hope that the work
presented in this thesis reflects the depth of my dedication and the
extent of my learning journey.

\newpage

\hypertarget{supervisor}{%
\section{Supervisor}\label{supervisor}}

Associate Professor Jan Christian Brønd, PhD

Research Unit for Exercise Epedimiology, Centre of Research in Childhood
Health, Department of Sports Science and Clinical Biomechanics,
University of Southern Denmark, 5230 Odense, Denmark

\hypertarget{assessment-committee}{%
\section{Assessment Committee}\label{assessment-committee}}

\hypertarget{chair}{%
\subsection{Chair}\label{chair}}

Associate Professor Anders And, PhD

\hypertarget{opponents}{%
\subsection{Opponents}\label{opponents}}

Professor Andersine And, PhD

Research Unit for Exercise Epedimiology, Centre of Research in Childhood
Health, Department of Sports Science and Clinical Biomechanics,
University of Southern Denmark, 5230 Odense, Denmark

Associate Professor Fætter Højben, PhD

Research Unit for Exercise Epedimiology, Centre of Research in Childhood
Health, Department of Sports Science and Clinical Biomechanics,
University of Southern Denmark, 5230 Odense, Denmark

\hypertarget{funding}{%
\section{Funding}\label{funding}}

The research presented in this thesis was generously funded by
TrygFonden, under grant numbers ID 130081 and 115606, and by the
European Research Council, under grant number 716657. Additional support
was provided by a one-year scholarship from the Faculty of Health
Sciences.

\newpage

\hypertarget{included-papers}{%
\section{Included Papers}\label{included-papers}}

\hypertarget{paper-i}{%
\subsection{Paper I}\label{paper-i}}

\begin{quote}
Manual Annotation of Time in Bed Using Free-Living Recordings of
Accelerometry Data\autocite{skovgaard_manual_2021}
\end{quote}

published in \href{https://doi.org/10.3390/s21248442}{Sensors}.

\hypertarget{paper-ii}{%
\subsection{Paper II}\label{paper-ii}}

\begin{quote}
Generalizability and Performance of Methods to Detect Non‑Wear with
Free‑Living Accelerometer
Recordings\autocite{skovgaard_generalizability_2023}
\end{quote}

published in \href{www.nature.com/scientificreports}{Scientific
Reports}.

\hypertarget{paper-iii}{%
\subsection{Paper III}\label{paper-iii}}

\begin{quote}
Improving Sleep Quality Estimation: A Comparative Study of Machine
Learning and Deep Learning Techniques Utilizing Free-Living
Accelerometer Data from Thigh-Worn Devices and EEG-Based Sleep Tracking
\end{quote}

Submitted to \href{https://www.nature.com/npjdigitalmed/}{npj Digital
Medicin}.

\hypertarget{english-summary}{%
\section{English Summary}\label{english-summary}}

bla

\hypertarget{danish-summary}{%
\section{Danish Summary}\label{danish-summary}}

bla

\hypertarget{overview-of-included-papers}{%
\subsection{Overview of Included
Papers}\label{overview-of-included-papers}}

The first paper is focused on improving the manual annotation of in-bed
periods by comparing the manual annotation method with established
EEG-based sleep monitoring devices and self-reported sleep diaries.

In the second paper, you delve into a more specific challenge in
physical activity sensor usage - the detection of non-wear. You propose
decision tree models that combine raw acceleration and skin temperature
data and emphasize the importance of external validation in
machine-learned models. \href{www.google.com}{check out google}

The third paper focuses on sleep quality estimation using machine
learning and deep learning models. You evaluate these models using data
from thigh-worn accelerometers, presenting a potential alternative for
large-scale sleep studies. You underscore the challenges of classifying
awake periods during in-bed time and the need for precision in assessing
individual sleep quality metrics.

There is a clear common thread that ties together all three of your
research papers. The overarching theme is the application of innovative
methods, particularly machine learning techniques, to improve the
utility, reliability, and accuracy of free-living accelerometer data in
monitoring human sleep and physical activity.

This common thread manifests in various aspects across your research:

\begin{enumerate}
\def\labelenumi{\arabic{enumi}.}
\item
  \textbf{Annotation of Time in Bed:} You worked on developing and
  validating a method for manually annotating in-bed periods in
  accelerometer data, which is essential for the accurate classification
  of sleep and other behaviors.
\item
  \textbf{Non-Wear Detection:} In this paper, you focused on a specific
  challenge with wearable activity sensors - the detection of non-wear
  time. You investigated machine learning-based decision tree models
  that leverage raw acceleration and skin temperature data for this
  purpose.
\item
  \textbf{Sleep Quality Estimation:} Lastly, you explored the use of
  both machine learning and deep learning models to estimate sleep and
  sleep quality metrics, specifically leveraging data from thigh-worn
  accelerometers, an underexplored area in the field.
\end{enumerate}

Throughout your work, there is a consistent emphasis on improving and
validating methods to better leverage accelerometer data for studying
human behaviors. This recurring theme suggests a significant
contribution to the field of sleep and physical activity research,
particularly in the context of large-scale studies where traditional
methods may be impractical or cost-prohibitive.

\hypertarget{introduction}{%
\section{Introduction}\label{introduction}}

\hypertarget{outline-of-introduction-section}{%
\subsection{Outline of Introduction
Section}\label{outline-of-introduction-section}}

\hypertarget{overview-and-background}{%
\subsubsection{Overview and Background}\label{overview-and-background}}

\begin{itemize}
\tightlist
\item
  The importance of sleep and physical activity tracking in health
  research.
\item
  The limitations of traditional methods, such as polysomnography and
  self-reported diaries.
\item
  The emergence and potential of wearable accelerometers and machine
  learning models in this field.
\end{itemize}

\hypertarget{scope-and-relevance}{%
\paragraph{Scope and Relevance}\label{scope-and-relevance}}

-The need for cost-effective, reliable, and practical alternatives for
large-scale studies. - The potential of free-living accelerometers, and
why they are a compelling subject of study.

\hypertarget{existing-challenges}{%
\paragraph{Existing Challenges}\label{existing-challenges}}

\begin{itemize}
\tightlist
\item
  Discuss the challenges with existing methods, such as identifying
  non-wear time, annotating in-bed periods, and classifying awake
  periods during in-bed time.
\item
  Address the lack of exploration of certain sensor locations, like the
  thigh.
\end{itemize}

\hypertarget{thesis-goals-and-objectives}{%
\paragraph{Thesis Goals and
Objectives}\label{thesis-goals-and-objectives}}

\begin{itemize}
\tightlist
\item
  Clearly state the aim and objectives of your thesis.
\item
  Explain how your thesis will address the identified challenges,
  including improving the manual annotation of in-bed periods, enhancing
  non-wear detection, and estimating sleep quality metrics.
\end{itemize}

\hypertarget{overview-of-the-papers}{%
\paragraph{Overview of the Papers}\label{overview-of-the-papers}}

\begin{itemize}
\item
  Briefly introduce each paper, highlighting the key research question,
  methods, and findings.
\item
  Explain how each paper contributes to your thesis goals and
  objectives.
\end{itemize}

\hypertarget{motivation-for-the-research}{%
\subsubsection{Motivation for the
Research}\label{motivation-for-the-research}}

\hypertarget{the-need-for-improved-annotation-techniques}{%
\paragraph{The Need for Improved Annotation
Techniques}\label{the-need-for-improved-annotation-techniques}}

\begin{itemize}
\tightlist
\item
  Importance of accurate annotation in accelerometer data analysis.
\item
  A brief discussion of the first paper's findings and implications.
\end{itemize}

\hypertarget{improving-non-wear-detection}{%
\paragraph{Improving Non-Wear
Detection}\label{improving-non-wear-detection}}

\begin{itemize}
\tightlist
\item
  Explain the implications of undetected non-wear time on data quality.
\item
  Highlight the findings of your second paper and its relevance.
\end{itemize}

\hypertarget{advancing-sleep-quality-estimation}{%
\paragraph{Advancing Sleep Quality
Estimation}\label{advancing-sleep-quality-estimation}}

\begin{itemize}
\tightlist
\item
  Discuss the impact of sleep quality estimation on understanding human
  sleep behavior.
\item
  Briefly describe the conclusions of your third paper.
\end{itemize}

\hypertarget{methodological-approaches}{%
\subsubsection{Methodological
Approaches}\label{methodological-approaches}}

\begin{itemize}
\item
  Give a brief overview of the methods used across all three studies,
  such as the use of machine learning models, deep learning techniques,
  manual annotation, and decision tree models.
\item
  Explain how these methods address the research objectives and the
  challenges identified earlier.
\end{itemize}

\hypertarget{thesis-structure}{%
\subsubsection{Thesis Structure}\label{thesis-structure}}

Provide an outline of the subsequent chapters of your thesis.

\hypertarget{test-header}{%
\section{Test Header}\label{test-header}}

Over the past decade, an expanding body of literature has underscored
the vital role of physical activity and sleep in sustaining overall
human health. These key components of our daily routine are not merely
elements of lifestyle; they are closely interwoven with our mental and
physical well-being, and disturbances in these areas can precipitate a
cascade of health issues. Our understanding of the dynamics of physical
activity and sleep and their implications on health, however, are
contingent upon the accuracy and precision of the methods we employ to
measure them. This raises a need for efficient, cost-effective, and
minimally invasive tools for longitudinal physical activity and sleep
monitoring.

In this context, body-worn motion sensors, specifically accelerometers,
have emerged as a significant advancement, providing valuable insights
into human physical activity and sleep patterns. Utilizing
accelerometers to measure and classify the intensity of human movement
and non-wear time has been an integral part of my research during my
Ph.D.~These devices offer high cost efficiency and minimal participant
burden, making them an ideal choice for large-scale studies. However,
the challenge lies in effectively interpreting the data generated by
these devices, particularly when it comes to accurately distinguishing
between wear and non-wear time.

Addressing this challenge has involved utilizing sophisticated machine
learning techniques to classify non-wear time in raw accelerometer data,
and my studies have extended to investigate the generalizability and
performance of these models on unseen data. As part of this effort, we
examined the potential benefits and limitations of machine learning in
detecting patterns from training data and approximating complex models,
taking care to balance model variance and bias. These findings
underscore the vital role of these technologies in enhancing the
accuracy and precision of non-wear time classification, and underscore
the importance of maintaining a balance between overfitting and
underfitting.

Further extending the utility of accelerometers, my research delved into
the domain of sleep science. Here, we utilized accelerometers as a
convenient and affordable means of tracking sleep patterns over extended
periods. Recognizing that sleep analysis has been dominated by wrist and
hip-worn accelerometers, our focus shifted to the less-explored, but
promising, realm of thigh-worn accelerometers for sleep assessment.
While there has been progress in estimating sleep duration using these
devices, the application of machine learning techniques in this area
remains underdeveloped, a gap that our research aimed to address.

We explored a range of machine learning and deep learning models,
utilizing raw data collected from a tri-axial thigh-worn accelerometer
to estimate in-bed and sleep time. We validated our models against an
electroencephalography-based (EEG) sleep tracking device, which served
as the gold standard for measuring sleep. Moreover, we evaluated the
models' performance in assessing crucial sleep quality metrics,
reinforcing the potential of accelerometers as a tool for comprehensive
sleep and physical activity tracking.

Taken together, this body of work sheds light on the immense potential
of wearable technology, particularly accelerometers, in revolutionizing
the field of human physical activity and sleep research. While
significant strides have been made in the development and validation of
advanced machine learning models to interpret accelerometer data, the
journey towards more accurate, user-friendly, and holistic methods of
tracking human physical activity and sleep continues. As I continue to
navigate this fascinating field, my research aims to bridge the gaps in
our understanding and contribute to a future where personalized health
monitoring is both accessible and precise.

\newpage


\printbibliography[title=References]


\end{document}
